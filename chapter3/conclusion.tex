This chapter has investigated how the current setup for core composition can be modified in order to improve performance.
Due to the fact that blocks are fetched in a serial fashion, when a large number of cores are composed this can severely reduce the amount of time that cores are actively executing blocks.
Therefore, finding a way of allowing cores to fetch in parallel can help increase the effectiveness of a composition.
However, another issue arises, the data-dependencies found in inter-block communication via register reads and writes can cause the serialisation of block execution.
Finding a way to predict data can potentially aleviate these data dependencies and thus further improve the performance of core composition.

This chapter proposed a round-robin fetching mechanism, where cores do not fetch sequential blocks, but rather fetch blocks in strides.
This enables cores to fetch independently from one another, without having to submit fetch requests to other cores.
Using such a scheme can potentially increase the performance of large core compositions on small blocks by a factor of 2 to 3x.

After covering the fetching scheme, the idea of using a block-based value predictor was covered.
This value predictor was initially designed by Perais. et al. in ~\cite{}.
The design choices behind the value predictor match some of the architectural features of EDGE, mainly to do predictions at the granularity of a block.

To understand how these two modification can impact performance, the same set of benchmarks used in Chapter~\ref{chp:cases} are explored here.
First, the performance of a perfect value predictor with the round robin fetching scheme is explored and shows that it can improve the performance of core composition by up to 5x, with an average of 1.9x.
This was followed by an analysis of the performance of using different configurations of the D-VTAGE value predictor.
Overall, using state-of-the art value prediction with round-robin fetching scheme leads to a performance improvement of up to 3x with an average of 1.5x.

This Chapter demonstrates that there are potentially 