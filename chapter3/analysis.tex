Overall, this chapter demonstrates that improving core composition from a hardware level requires the implementation of complex techniques.
Whilst a data predictor improves performance by allowing blocks to execute independently via data speculation, implementing a data predictor in hardware is considered difficult~\cite{peraisBeBop2015}.
Even so, without adding extra features to the hardware to support core composition, performance can be stunted when the software does not generate the ideal block.
This is a similar problem to early VLIW and superscalar machines (cite).

With this in mind, the results shown in Section~\ref{sec:chp3:res} do demonstrate that if these techniques are implemented in hardware, it can lead to important performance improvements.
These performance improvements are due to the reduced overhead of populating the core composition with blocks and the register dependencies being resolved via speculation.
The section even shows that small data predictors can make a difference and that only targetting register reads is a viable solution.