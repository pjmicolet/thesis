In EDGE, block dependencies are expressed through register reads and writes.
To ensure that blocks in flight do not execute a register read before an older block has executed a write to the same register, register writes are encoded in the header.
This informs the register-file which can then effectively push back speculative reads from younger blocks.
Whilst this ensures correct execution of speculative blocks, it effectively reduces the potential for block level parallelism (BLP).
This is further exacerbated when composing cores is considered, as this increases the amount of blocks that may potentially have to wait on register reads and writes.
For example, tightly knit loops that write most values out to registers cannot be optimised using core composition due to register dependencies.

In situations where register dependencies are a bottleneck and cannot be optimised via a compiler, core composition cannot be considered an effective method of improving performance.
The problem of trying to reduce register and memory dependencies to improve instruction level parallelism is not new, and is an issue for more traditional superscalar processors.


In these cases value prediction can be used to attempt to ensure that these blocks can run in parallel even if there are dependencies.
