Dataflow programming languages facilitate the design of data intensive programs such as streaming applications commonly found in embedded systems.
They also expose parallelism that can be exploited using multicore processors which are now part of the mobile landscape.
In recent years a shift has occurred towards heterogeneity (\eg ARM big.LITTLE) and reconfigurability.
Dynamic Multicore Processors (DMPs) bridge the gap between fully reconfigurable processors and homogeneous multicore systems.
They can re-allocate their resources at runtime to create larger more powerful logical processors fine-tuned to the workload.

Unfortunately, there exists no accurate method to determine how to partition the cores in a DMP among application threads.
Often programmers rely on analysing the application manually and using a set of hand picked heuristics.
This leads to sub-optimal performance, reducing the potential of DMPs.
What is needed is a way to determine the optimal partitioning and grouping of resources to maximise performance.

As a first step, this paper studies the effect of thread partitioning and hardware resource allocation on a set of StreamIt applications.
We show that the resulting space is not trivial and exhibits a large performance variation depending on the combination of parameters.
We introduce a machine-learning based methodology to tackle the space complexity.
Our machine-learning model is able to directly predict the best combination of parameters using static code features.
The predicted set of parameters leads to performance on-par with the best performance found in a space of more than 32,000 configurations per application.
