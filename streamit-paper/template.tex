\documentclass[pageno]{jpaper}

%replace XXX with the submission number you are given from the ASPLOS submission site.
\newcommand{\asplossubmissionnumber}{XXX}

\usepackage[normalem]{ulem}

\begin{document}

\title{}

\date{}
\maketitle

\thispagestyle{empty}

\begin{abstract}     
    The emergence of dynamic multicore systems allows for new hardware
    optimisations where hardware modifies itself to match the software's needs.
    In this paper we describe an automatic method of determining how to
    partition a program and associate each partition to a set of fused cores in
    order  to improve the performance of streaming applications through the use
    of machine learning. We explore the difficulties of extracting important
    features out of programs in order to accurately decide the correct chip
    topology.  Using the methods described in the paper we can achieve up to 2
    times speedup on certain applications.
  
\end{abstract}

\section{Introduction}
% Software point of view: the problem
A Dynamic Multicore Processor's (DMP) ability to reconfigure itself allows it to adapt to any program it executes.
Whilst being able to reconfigure hardware is a promising approach to optimising execution, DMPs come with their own set of challenges when attempting to finding a good configuration for the program at hand.
Given a program that can be parallelised, a DMP can either be configured to run a high amount of threads on small groups of cores, a small number of threads on large groups of cores or a heterogeneous mix of both large and small cores.
Without deep knowledge of the architecture, knowing how to configure the processor correctly, to be able to obtain the best performance, can be a highly time consuming task.
This can be further complicated if the programming model does not provide any insights on how the program may be partitioned into threads.
The problem of optimising multithreaded software for DMPs can therefore be split into two distinct tasks.
First, finding a programming model that makes software partitioning into threads explicit.
Second, using information from both the hardware and software, automate the partitioning of both the software into threads, and the hardware into logical cores.

In most parallel programming models such as OpenMP~\cite{openmp}, the user is directly responsible for mapping parallelism to the hardware; a difficult and time consuming task~\cite{prabhu2011LanguagePar}.
This is due to the fact that these models extend programming languages that do not consider parallelism as a defining design factor~\cite{pingaliTao2011}.
On the other-hand dataflow programming models such as StreamIt~\cite{theis2002streamit} and Lime \cite{auerbach2012lime} make data and paralleism a first class citizen.
In dataflow languages, applications are expressed as data oriented graphs and --- ideally --- the compiler or runtime determines the mapping of parallelism onto the available hardware and controls the grouping of hardware resources.
Thus using such a model can be a potential solution to the first part of the problem.

However, optimally mapping parallelism and managing hardware resources remains an open problem given the sheer complexity of the resulting design space.
For example, given a 16 core DMP with up to 15 threads, a single program can have over 32,000 different configurations of thread to logical-core pairings.
Rather than exhaustively searching the space, which is a very time consuming task, finding a way to automate the configuration of the processor makes the use of DMPs more attractive.
The number of program features that may influence how to partition dataflow programs is large, for example it could depend on the number of tasks, the parallelism made explicit by the language and/or different compiler optimisations.
Therefore manually determining a set of heuristics to create a model that selects thread count and core configurations is not recommended as important information may potentially be disregarded.
Instead, a machine learning model paired with correlation analysis, can extract all the features that correlate the most with deciding a good partition and generate an appropriate model.

This chapter analyses how static ahead-of-time reconfiguration of a dynamic multicore processor can improve performance of a set of streaming applications.
These streaming applications include audio signal processing algorithms, image processing algorithms and sorting algorithms.
Streaming programs are ubiquitous in the embedded systems space~\cite{theis2002streamit} and their mix of parallelism and computation make them an interesting domain for DMPs.

An analysis of the design space exploration is performed and shows the impact of modifying resources and thread mapping.
This analysis is conducted using a set of StreamIt programs that are ran on a verified cycle-level simulator for a tiled reconfigurable architecture with support for core composition.
A machine learning model is developed using the information gathered from the exploration.
This model predicts the best number of threads for a given application and an optimal number of cores to allocate to each thread.

To demonstrate the viability of the approach the results of the predictive model are compared to the best sampled thread and core composition pairing in a space of more than 32,000 design points.
The model can match the performance of the best sampled points in the space, with speedups of up to 9x on a 16 core processor compared to single threaded execution on a single core. 

% contributions
The main contributions of this chapter are:
\begin{itemize}
\item An analysis of the co-design space of thread partitioning and core composition;
\vspace{-1em}
\item A study on the impact of a loop transformation on the optimal core composition;
\vspace{-1em}
\item A machine-learning model to determine the optimal core composition and thread partitioning ahead of time in order to get the optimal performance;
\vspace{-1em}
\item An analysis of the most static code features that are considered the most important for determining a correct configuration of the system by the model.
\end{itemize}


The rest of the chapter is structured as follow.
Section~\ref{sec:motiviation} motivates this work by showing the complexity of the design space.
Section~\ref{chp:stream:sec:setup} describes the methodology and section~\ref{sec:streamit:dse} presents an in-depth analysis of the design space.
Section~\ref{sec:ml} develops a machine-learning model to predict the best thread mapping and core composition while Section~\ref{sec:results} shows the performance achieved by the model.
Section~\ref{sec:conclusion} concludes this chapter.

%In most parallel programming models such as OpenMP, the user is directly responsible for mapping parallelism to the hardware; a difficult and time consuming task.
%This problem is further exacerbated when hardware resources can be combined since programmers have to take into account the dynamic behavior of the architecture~\cite{bower2008impactd}.

% Solution for the software: data flow programming
%To solve this problem, this chapter demonstrates that there is a need to raise the programming abstraction and remove the burden of mapping parallelism from programmers.
%Dataflow programming models such as StreamIt~\cite{theis2002streamit} and Lime~\cite{auerbach2012lime} offer one part of the solution.
%Applications are expressed as dataflow graphs and --- ideally --- the compiler or runtime determines the mapping of parallelism onto the available hardware and controls the grouping of hardware resources.
%However, optimally mapping parallelism and managing hardware resources remains an open problem given the sheer complexity of the resulting design space.

% What we do: 1st an analysis

\section{Motivation}
%\begin{figure}[h]
%    \centering
%    \includegraphics[width=0.7\textwidth]{streamit-paper/graphics/beamformer_motivation.pdf}
%    \caption{Distribution of the runtime for Beamformer resulting from an exhaustively exploration of the hardware/software co-design space.
%     The application has been partitioned into different number of threads and core compositions.}
%     \label{fig:beamformermotiv}
%\end{figure}
\subsection{Finding an optimal configuration}
This section motivates the difficulty of finding a good combination of thread and core partitioning.
First, a simple experiment is conducted where the \bench{FilterBank} StreamIt benchmark is analysed using a 16 core dynamic multicore processor.
%Maybe reform this sentence
\bench{FilterBank} distributes its inputs amongst an array of discrete fourier transform (DFT) filters, the outputs of the DFT filters are down-sampled, up-sampled and then recombined to form a processed signal~\cite{streamitrepo}.
The program's tasks are partitioned into threads and a various number of cores are allocated to each of the threads.
Since one of the threads is the master which creates and joins all worker threads, this means that the application can be partitioned in up to 15 threads, and 15 cores can be used for those threads.
As each thread must have at least one core assigned to it, and not all cores have to be grouped together, the total number of configurations are:

\begin{equation}
15 + \sum_{threads=2}^{15} \bigg( \sum_{cores=threads}^{15} \frac{(cores-1)!}{(threads-1)!((cores-1)-(threads-1))!}\bigg)
\label{eq:comb}
\end{equation}

In Equation~\ref{eq:comb}, the constant 15 represents the 15 different number of cores that can be composed for the single threaded version.
There are 32,767 combinations (exhaustive space) of thread mappings and core compositions pairings that can be generated.
In this chapter, a design point represents one of these 32,767 different configurations.

\begin{figure}[t]
    \centering
    \includegraphics[width=1\textwidth]{streamit-paper/graphics/filterbank_motivation_4.pdf}
    \caption{Distribution of speedups for FilterBank when using different core composition and thread pairings compared to single-core single thread. The dots on the X-axis represent specific configurations.}
     \label{fig:threadcoremotiv}
	 \vspace{-1em}
\end{figure}

Figure~\ref{fig:threadcoremotiv} presents the speedup distribution from a subset of the co-design space of \bench{FilterBank} as a density graph.
The speedup is measured by comparing the performance of the design point to that of a single-core/single thread.
A single-core/single thread is used here as it represents the default "unmodified" configuration.
For this experiment, 1316 different thread/core combinations are explored; the reason this number is chosen is explained later on in section~\ref{chp:stream:sec:setup}.

As can be seen in figure~\ref{fig:threadcoremotiv}, the majority of the design points result in a speedup of 2.7x.
The best speedups however are far fewer than the average case (4x smaller density) and can improve performance by up to almost 6x.
Thus, if a good configuration is found, this can yield an important speedup compared to the average configuration.
However since the number of good configurations is low, this underlines the notion that finding a combination of threads and cores is a non-trivial endeavor, as randomly choosing a configuration will result in a sub-optimal performance.
Indeed, even if the average case is 2.7x faster than a single core, Figure~\ref{fig:threadcoremotiv} shows that there exists a good number of configurations that can lead to less than average speedups.

\subsection{Minimizing the search space}
Whilst there exists a large variety of thread-core combinations, certain design choices can be used to try and minimise the space.
For example, choosing to only do multithreading reduces the search space to 15 possible solutions whilst only combinations that lead to homogeneous core compositions reduces the search space to:

\begin{equation}
\sum_{threads=1}^{15} \lfloor\frac{15}{threads}\rfloor= 45
\end{equation}

where the constant 15 represents the number of cores available.
Using only homogeneous core compositions, which facilitates the core partitioning decision, would therefore lead to 45 possible solutions.
However, reducing the search space limits the potential obtainable speedup.
Figure~\ref{fig:threadcoremotiv} shows the performance distribution of these specific design points.
Their location on the X-axis represents the speedup obtained for that specific configuration.
The points represent a set of different design choices such as only using multithreading, using homogeneous core-compositions with threads and using heterogeneous core-compositions with threads.

Using only multithreading can lead to some performance improvements, however it will not result in the optimal performance.
For the \bench{FilterBank} benchmark, 4 threads leads to the fastest execution time when only using multithreading.
This performance is in the highest peak of the density curve, which means that finding the best number of threads for the benchmark will only lead to average performance improvements in this case.
However, using too many threads, such as 15 threads, leads to a degraded performance compared to the average.
This is due to the fact that the communication overhead between threads will be too high.

To explore homogeneous core composition, the optimal number of threads, which is the number of threads that leads to the fastest execution time without core-composition as a baseline is used.
In this case only 2 homogeneous pairings exist: 2 cores fused for each of the 4 threads or 3 cores fused for each of the 4 threads.
Figure~\ref{fig:threadcoremotiv} shows that homogeneous core-composition will outperform only using multithreading by 1.3x, however it is not the optimal solution.
In the end, using a heterogeneous configuration leads to a 1.5x speedup compared to the fastest homogeneous configuration.
Therefore, it is important to consider all possible configurations to ensure the possibility of obtaining the best performance.

\subsection{Summary}
This section shows that multi-threading with heterogeneous core-composition is the optimal solution.
This means that the total space must be explored in order to ensure that the best speedup can be found.
Due to the size of the space and the fact that there can be no apriori about good configurations, using machine learning to predict configurations is a promising approach. % be used to predict good configurations.
By exploring a set of StreamIt benchmarks a machine learning model can learn what features correlate with the correct configuration.
This example illustrates the necessity for designing the technique to predict both the optimal number of threads and core composition to use.
The next section will present a more in-depth analysis of the design space before presenting our machine-learning predictive model.


\section{Background}
\chapter{Background}
This chapter covers the different topics that are present in this thesis.
The background starts by briefly covering Chip Multicore Processors and Heterogeneous Chip Multicore Processors to motivate the existence of Dynamic Multicore Processors.
Then the core-fusion technique, which is the main mechanism brought forward by Dynamic Multicore Processors is described in detail.
This is followed by a description of the EDGE instruction set architecture which is used in the Dynamic Multicore Processor described in this thesis.
Finally, streaming programming languages, which are used in Chapter~ref{}, are explained.

\section{Chip Multicore Processors}

\begin{figure}[t]
 \center
 \includegraphics[width=1\textwidth]{background/graphics/i7intel.jpg}
 \caption{Intel Core i7 processor internal die photograph taken from intel whitepaper}\label{fig:i7}
\end{figure}
 
Chip Multicore Processors (CMPs) have become ubiquitous due to the difficulty in scaling single core performance.
In a CMP, multiple processor cores are put on a single package as can be seen in Figure~\ref{fig:i7}.
The most common CMP uses homogeneous cores as they reduce the design complexity both from a hardware and software perspective~\cite{}.
Unlike single core systems, the performance improvement in CMPs come from running multiple tasks in parallel.
These tasks can either be different programs or multiple threads from the same program running on the multiple cores.
By defining speedup \textit{S} to be the original execution time of the program over the new execution time with \textit{n} processors and \textit{f} representing the fraction of the program which can be parallelised; Amdahl's Law states

\begin{equation}
S = \frac{1}{(1-f) + \frac{f}{n}}
\end{equation}\label{amdlaw}

thus, given an infinite number of processor cores~\cite{ekhout2010amdalh}

\begin{equation}
\lim_{n\to\infty} S = \frac{1}{(1-f)}
\end{equation}

This second equation demonstrates how, given any program, the speedup obtained by using a CMP will be limited to the fraction \textit{f} of parallel code found in the program itself.
As all the processor cores are homogeneous this will cause serial bottlenecks to severely reduce the potential speedup as no core is adapted to speedup such regions.
This implication has pushed research into finding ways of parallelising code to its fullest~\cite{}, however this may not always be possible~\cite{}.
Thus whilst CMPs have become a mainstain in processor design, the homogeneous model has its limits.

\section{Heterogeneous Chip Multicore Processors}

\begin{figure}[t]
 \center
 \includegraphics[width=1\textwidth]{background/graphics/biglittle.png}
 \caption{Example of a heterogeneous multicore processor proposed by ARM (big.LITTLE)}\label{fig:blarm}
\end{figure}

Unlike CMPs, Heterogeneous Chip Multicore Processors (HCMPs) or Asymmetrical Chip Multicore Processors (ACMPs) bring a variety of cores onto a single package.
This may come in different forms, such as having multiple instruction set architectures on the same system on chip (MPSoCs)~\cite{venkatharnessingISA,hipstr}, or same ISA different size cores on an SoC~\cite{bigLittle}.
For example, Figure~\ref{fig:blarm} shows a schemata for ARM's big.LITTLE HCMP, where a high-performance Cortex-A15 is paired with a simpler, power efficient Cortex-A7.
The two cores are connected via a cache coherent interconnect which provides data coherence at the bus-level, allowing the cores to make reads to its neighbor.
Software is then run on one of the cores depending on a profile; if the user requires performance over energy, then the Cortex-A15 will be chosen, however if energy/power efficiency is required then the Cortex-A7 will be chosen.

This small example already demonstrates an advantage of HCMPs; unlike CMPs, the variety of cores on an HCMP provide a flexibility to the hardware.
This can be used for different purposes, such as security~\cite{hipstr}, energy/power savings~\cite{venkatharnessingISA} and speeding up applications~\cite{venkatharnessingISA}.
In their 2014 paper, Venkat et al.~\cite{venkatharnessingISA} demonstrate that a multi-ISA HCMP can improve performance by up to 1.4x and achieve energy savings of up to 40\% compared to a CMP on a peak-power budget of 40W.
They motivate the idea that HCMPs with heterogeneous ISAs even improve over the performance of single-ISA HCMPs with speedups around 15\% and energy savings of 21.5\%.

However, whilst the hardware diversity in HCMPs is an advantage compared to CMPs, it also increases programming complexity.
For example, Gupta et al. in ~\cite{dypo} show that a single-ISA octa-core big.LITTLE architecture can have 20 CPU cores, combined with the ability to dynamically modify the voltage, this leads to 4000 different configurations.
This highly increases the complexity of obtaining the correct settings for different programs.
MPSoCs also face a similar issue as having more than a single ISA not only adds design challenges, but program migration between different cores may in fact deteriorate performance~\cite{asplos2012}.

\section{Dynamic Multicore Processors}

% This section explains what a dynamic multicore is

In both CMPs and HCMPs, once the chip is fabricated the design cannot be modified, meaning that many of the trade-offs between power, performance and area cannot be changed after production.
Dynamic Multicore Processors (DMPs) attempt to bridge the gap between the two previous designs by allowing the execution substrate to adapt dynamically at runtime.
Mitall's survey ~\cite{MittalSurv2016} defines three types of modifiable resources: the core count~\cite{ipek2007CoreFusion}, number of resources that each core has~\cite{Homayoun3DPooling2012} and microarchitectural features~\cite{fallinhetblock2014,BauerRSE08,tavanaElastic}.

\subsection{Core Fusion Dynamic Multicore Processors}

\begin{figure}[t]
    \centering
    \includegraphics[width=.7\textwidth]{streamit-paper/graphics/dmcgraph.pdf}
    \caption{High-level view of a dynamic multicore processor that can modify its core count.}
    \label{fig:dynmulticore}
\end{figure}

A DMP that modifies core count is composed of homogeneous cores with a reconfigurable fabric.
Physical cores can function either on their own or as a group of physical cores; this is called a Logical Core (LC).
Throughout this thesis, the term core-fusion will be used to define the mechanism of cores creating an LC.
A logical core will fetch instructions from a single source and execute them accross all the physical cores that compose the LC.
Cores can fuse dynamically and create a logical core of any sizes.
For example in Figure~\ref{fig:dynmulticore}, the DMP fuses cores into 3 LCs of sizes 1, 8 and 6 physical cores.
The exact mechanism of core-fusion are described later on in Section~\ref{sec}.

The advantage of a core-fusion DMP over the traditional CMP or HCMP is the ability to reconfigure the processor dynamically to better match the tasks at hand.
For example, large sequential sections of code with high Instruction Level Parallelism (ILP) can be accelerated on a logical core that mimics a wide superscalar processor.
On parallel workloads the DMP can be reconfigured by de-composing the logical cores as seen in Figure~\ref{fig:dynmulticore} to match the Thread Level Parallelism (TLP).

%More here
\subsection{Resource Sharing Dynamic Multicore Processors}
A more fine-grained reconfiguration can be found in resource-sharing DMPs.
There exist different models for resource sharing DMPs.
For example the WiDGET DMP by Watanabe et al.~\cite{watanabe}, cores are built out of Instruction Engine front-ends which function similarly to Out of Order (OoO) cores' front and back ends.
They then are connected to Execution Units which they can choose to use.
Each core in the WiDGET DMP also have access to their neighbors Execution Units, allowing for more variation.
Another example of resource sharing can be found in Rodrigues et al.'s work~\cite{} where a core can use resources such as Arithmetic Logic Units (ALUs) from other cores.

%More here
\subsection{Microarchitectural Reconfigurable Dynamic Multicore Processors}
A final example is a DMP which can reconfigure microarchitectural features to better fit the current application.
Fallin et al.~\cite{fallin} observe that serial code can exhibit phases that fit different microarchitectural features.
According to them, these phases may only been in the ten to hundred thousands instructions long.
These DMPs can therefore modify microarchitecural features, such as in-order or out-of-order execution, to best match the current phase of a program.

\begin{figure}[t]
    \centering
    \includegraphics[width=1\textwidth]{background/graphics/EDGE_3.pdf}
    \caption{High-level view of the EDGE ISA flow.}
    \label{fig:EdgeHigh}
\end{figure}

\section{EDGE Instruction Set Architecture} 
The Explicit Data Graph Execution~\cite{burger04edge} (EDGE) instruction set architecture (ISA) is a data-flow based ISA.
Figure~\ref{fig:EdgeHigh} shows a high-level overview of how EDGE differs from a traditional instruction set architecture.
The EDGE compiler has a first pass which generates instructions from the original source code.
Blocks are then generated from the basic-blocks found in the code generation pass.

%what I'm trying to say here is that registers are used only for outer communication, in a block instructions are 
Unlike traditional ISAs, blocks do not communicate via registers, but rather the output targets of instructions are encoded to instruction inputs~\cite{smith2006edge}.
Loads and stores in each EDGE block are assigned unique identifiers which are used resolve load-store dependencies.
Thus, the EDGE ISAs encode dependencies between instructions at the ISA level, registers are only used for inter-communication between blocks.
An EDGE block also contains a header that will inform the hardware about the number of stores and register writes contained in the block~\cite{e2paper}, this is used to facilitate committing blocks.

EDGE blocks also have a set of restrictions to satisfy correctness.
If a block does not meet these requirements, it may need to be broken down into smaller blocks.
These restrictions are:

\begin{itemize}
\item Block Size: an EDGE block may be between 4 to 128 instructions.
\item Load/Store: an EDGE block may have at most 32 load/store instructions.
\item Entry/Exit: an EDGE block may have a single exit but may have multiple exits.
\end{itemize}

To increase the average size of EDGE blocks, multiple blocks can be combined together to form one large block called a hyperblock.
This is achieved through the use of instruction predication.
For example given an if/else statement, the compiler can generate a single block, predicating all instructions of the else statement.
As the compiler needs to declare the number of stores and register writes in the block header, extra instructions may need to be generated to ensure the block always executes the same amount of stores.

Overall, the EDGE ISA enables the architecture to dispatch blocks speculatively, with low overhead~\cite{putnam2010e2,kim2007tflex}, therefore, increasing exploitation of ILP.

\section{EDGE Processor}

\subsection{Core Lanes}
 \begin{figure}[t]
 \center
 \includegraphics[width=1\textwidth]{background/graphics/e2segment.png}
 \caption{Example of a four lane core on an EDGE processor taken from~\cite{e2smith}.}\label{fig:e2segment}
 \end{figure}
 
EDGE instruction blocks can be up to 128 instructions long, however this often isn't the case.
To maximize core-utilisation, each core on an EDGE Processor is segmented into a set number of lanes which can each fetch and decode their own blocks.
A lane is able to fetch a block of maximum size

\begin{equation}
\frac{128}{NumberOfLanes}
\end{equation}

For example, a four lane core as seen in Figure~\ref{fig:e2segment} can have up to four blocks of 32 instructions.
Fetching blocks larger than 32 instructions will fill up more than one lane.
Lanes allow EDGE cores to be more flexible to block size variability.

\subsection{Core Fusion}
 \begin{figure}[t]
 \center
 \includegraphics[width=1\textwidth]{cases-paper/graphics/background/proc_test.pdf}
 \caption{Core Fusion Mechanisms for our EDGE-based architecture.}\label{fig:dmp}
 \end{figure}
 
Core Fusion is achieved by fusing a set of \textit{physical} cores to create larger \textit{logical} cores.
This does not modify the physical structure of the chip, instead it provides a unified view of a group of physical cores to the software.
In the processor used throughout the thesis, the micro-architecture is distributed: register files, Load Store Queues (LSQs), L1 caches and ALUs all look like nodes on a network.
This means that when cores fuse together, this is similar to adding an extra node to the network.
Fusion is a dynamic modification and may occur during the execution of a program to better fit the workload.
Unlike traditional CMPs, fused cores will operate on the same thread and attempt to extract Instruction Level Parallelism (ILP) rather than Thread Level Parallelism (TLP)~\cite{micolet2016dmpstream,pricopi2012bahurupi}.

Figure~\ref{fig:dmp} shows the different stages and mechanisms of core fusion for a four core system.
When creating a new core fusion a master core informs all other cores about the fusion and sends the predicted next block address to the next available fused core.
When we start a new thread on a fused core the OS and runtime write the new core mapping to a system register.
The hardware then flushes these cores if they are not idle and sets the PC of the first block of that thread on one core in the logical processor and starts executing.
When a core mispredicts a branch in a fusion, it informs the other cores which flush any younger blocks.
When un-fusing, the master core informs the other cores, which then commit or flush their blocks and power down while the master core continues to fetch and execute blocks from the thread.
The extra hardware required to support dynamic reconfiguration is very minimal~\cite{kim2007tflex} since most of the machinery already in place can be reused such as the cache coherence protocol when fusing and un-fusing the cores.

When a logical core fetches multiple blocks, it may execute them out of order.
However memory instructions and instructions that modify registers pass through the LSQ and register-file and are executed in order.
This ensures that blocks operate on memory in a consistent fashion.
In case of a memory violation caused by undetected dependencies a flush of all blocks younger than the violator, including the violating block, is performed.

%Fusing cores is therefore a lightweight process.
%We estimate that switching the size of the logical-core (LC) results in a delay of 100 cycles on average.
%The actual time varies based on the time it takes the cache coherence protocol to move the data around the memory system.
%Section~\ref{sec:reconfoverhead} discusses in more details how latency affects energy efficiency and shows that dynamic core fusion is still highly beneficial even when considering overheads of 1,000 cycles.

\section{Streaming Programming Languages}

% % This section should explain what steaming programming is (remove all the details about each language)
% General purpose programming languages often propose very little support for programs that handle with a continuous flow of data.
% This results in having to design a set of complicated for loops to manage the streams of data.
% Having to deal with different rates of incoming and outcoming data also increases the complexity of writing these applications using a standard language.

Streaming programming languages are a branch of dataflow programming that focus on applications that deal with a constant stream of data.
These applications, such as audio or video decoding can be commonly found in mobile devices.
Unlike conventional programming languages such as C++, these languages abstract the concept of incoming and outgoing data to permit the programmer to focus on how the data should be treated.
Programs are described as directed graphs where nodes are functions and their edges represent their input and output streams. 
These languages offer primitives to describe such a graph~\cite{theis2002streamit} which expose parallelizable and serial sections of the application directly to the compiler. 
Rates of incoming and outcoming data can also be defined to facilitate load balancing optimizations~\cite{chen2005rawstream}.

Features of streaming programming languages make them an ideal language for targeting multicore processors.
The explicit data communication between the different tasks in the program, the ability to estimate the amount of work performed in each task and information about data rates between tasks allows the compiler to easily generate a multi-threaded application that can run on a dynamic multicore processor.
However, the main challenge consists of deciding how to map the different tasks onto threads and how to allocate the right amount of resources to maximize performance.

\section{Machine-learning guided optimisations}


\section{Experimental Setup}

This section presents the design exploration of a set of streaming applications being executed on a DMP.
The section describes how changing the thread mapping and core composition affect the benchmarks and what can be learned from this.
In addition, the impact of loop unrolling and how it helps exploit larger fused cores is investigated.


\begin{figure}[t]
    \centering
    \includegraphics[width=1\textwidth]{streamit-paper/graphics/explanation3.pdf}
    \caption{Description of the workflow.
    Two distinct machine-learning models are used to predict the optimal thread partitioning and core composition based on static code features.}
    \label{fig:overview}
\end{figure}

\subsection{Overview}

\begin{figure}[t]
    \centering
    \includegraphics[width=1\textwidth]{streamit-paper/graphics/examplestrem.pdf}
    \caption{Example of a StreamIt program being partitioned into threads (represented by the different colours) followed by assigning cores to each thread.}
    \label{fig:examplestream}
\end{figure}

Figure~\ref{fig:overview} presents the workflow of the system used in this chapter and Figure~\ref{fig:examplestream} illustrates the workflow on a synthetic StreamIt graph.
First, the source-to-source StreamIt compiler is used to unroll loops as this is often beneficial when cores are composed as will be seen later in Section~\ref{sec:streamit:dse}.
Then, static code features such as the program's graph structure are extracted from the StreamIt code through the StreamIt source-to-source compiler.
These features are used as an input to the first machine-learning model that determines how many threads will be required based on an estimate of Thread Level Parallelism (TLP) found in the program.
This information is used to partition the program into threads which is done by the StreamIt compiler which produces a C++ program using pthreads.
This is exemplified in Figure~\ref{fig:examplestream}, the colours filling in the nodes represent the threads each node has been assign to.
This C++ program is then compiled using the compiler for EDGE described in Chapter~\ref{chp:setup}.

Then, a second machine-learning model is deployed which also analyzses static code features extracted from the SteamIt code, once again provided by the source-to-source compiler.
This model decides on the number of cores each thread will have.
This is achieved by estimating the amount of Instruction Level Parallelism (ILP) that can be possibly extracted in each thread and by determining how many physical cores should be fused for that thread.
Finally, the processor is reconfigured to fuse the requested resources ahead of time and execute the partitioned program.
For example in Figure~\ref{fig:examplestream}, once the graph is coloured, the machine learning model will estimate the potential ILP in each group of coloured nodes and then assign a number of cores each thread will execute on.

\subsection{Design Space}

This chapter explores 15 StreamIt benchmark all taken from the official StreamIt repository~\cite{streamitrepo}.
These applications represent a variety of embedded applications and kernels, from digital signal processing to a matrix-multiplication kernel or band pass filters.
There exist other benchmarks in the repository, however at the time of writing these benchmarks did not execute correclty on the provided simulator.

Table~\ref{tab:instancefilt} shows the number of filter instances and SplitJoins for each of the benchmarks.
As a refresher from Chapter~\ref{chp:Background}, SplitJoin filters are functions which distribute and collect data from parallel filters.
The applications feature a different number of SplitJoins which determine the task-level parallelism.
This is to include a variety of situations to test the flexibility of the dynamic multicore processor.
Whilst SplitJoins often facilitate the decision of how to partition the programs into threads, they are not the only way to exploit thread level parallelism.
The applications which do not feature SplitJoins can still be split into threads and will operate in a pipelined fashion~\cite{theis2002streamit}.
For each benchmark the default inputs provided in the repository~\cite{streamitrepo} are used and the default iteration count is set to 10. 

\begin{table}[t]
% The FFT need are variable
  \small
 \begin{tabular} { | l | l | l | l | l | l | }
 \hline
 \cellcolor[gray]{0.7}Type  & \cellcolor[gray]{0.7}Audiobeam&  \cellcolor[gray]{0.7} Beamformer& \cellcolor[gray]{0.7}Bitonic-Sort  &  \cellcolor[gray]{0.7} BubbleSort &  \cellcolor[gray]{0.7}  CFAR\\ \hline
  Filter Instances & 18 & 56 & 82 & 18 & 3 \\ \hline
	\# of SplitJoins &	1 & 2 & 44 & 0 & 0 \\ \hline

 \cellcolor[gray]{0.7}Type  & \cellcolor[gray]{0.7}ChannelVocoder &  \cellcolor[gray]{0.7} FFT&  \cellcolor[gray]{0.7}FFT3 &  \cellcolor[gray]{0.7} FFT6&  \cellcolor[gray]{0.7}FilterBank \\ \hline
  Filter Instances & 53 & 20 & 185 & 99 & 67 \\ \hline 
   \# of SplitJoins &	 1 & 12 & 44 & 96 & 9 \\ \hline 

   \cellcolor[gray]{0.7}Type& \cellcolor[gray]{0.7}FIR &  \cellcolor[gray]{0.7} FMRadio &  \cellcolor[gray]{0.7} InsertionSort &  \cellcolor[gray]{0.7} Matmul-Block &  \cellcolor[gray]{0.7} RadixSort\\ \hline
  Filter Instances& 131 & 29 & 6 & 4 & 13 \\ \hline
  \# of SplitJoins&    0 & 7 & 0 & 7 & 0 \\ \hline

 \end{tabular}
  \caption{Number of filter instances and SplitJoin filters present in each benchmark.}\label{tab:instancefilt}
\end{table}

%\begin{table}[t]
% The FFT need are variable
 % \small
 %\begin{tabular} { | l | l | l | l | l | }
 %\hline
 %  \cellcolor[gray]{0.7}Audiobeam&  \cellcolor[gray]{0.7} Beamformer& \cellcolor[gray]{0.7}Bitonic-Sort  &  \cellcolor[gray]{0.7} BubbleSort &  \cellcolor[gray]{0.7}  CFAR\\ \hline
 % 1 & 2 & 44 & 0 & 0 \\ \hline
 %  \cellcolor[gray]{0.7}ChannelVocoder &  \cellcolor[gray]{0.7} FFT&  \cellcolor[gray]{0.7}FFT3 &  \cellcolor[gray]{0.7} FFT6&  \cellcolor[gray]{0.7}FilterBank \\ \hline
 % 1 & 12 & 44 & 96 & 9 \\ \hline 
 %  \cellcolor[gray]{0.7}FIR &  \cellcolor[gray]{0.7} FMRadio &  \cellcolor[gray]{0.7} InsertionSort &  \cellcolor[gray]{0.7} Matmul-Block &  \cellcolor[gray]{0.7} RadixSort\\ \hline
 % 0 & 7 & 0 & 7 & 0 \\ \hline
% \end{tabular}
%  \caption{Number of split joins present in each benchmark.}\label{tab:splitjoin}
%\end{table}

\begin{table}[t]
\centering
\begin{tabular} { p{5.2cm}  p{1.8cm} }
      \toprule
      \textbf{Parameter} & \textbf{Values} \\ \midrule
      \# of cores in the processor & 16 \\
      \# threads per application & 1 -- 15 \\
      \# cores per thread & 1 -- 15 \\ \midrule
      \# sampled core compositions & 100 \\ 
      \# our sampled space & 1316 \\
      \# total sample space & 32767 \\ \bottomrule
    \end{tabular}
  \caption{Design space considered per application.}
  \label{tab:space}
\end{table}

The parameters and size of the space are given in Table~\ref{tab:space}.
In this study a 16 core DMP is used.
Having 16 cores allows for a larger variety of configurations, this also represents a processor similar to a tiled embedded system such as Tilera or Raw.
All cores in the DMP are utilised; Core 0 is assigned to the main thread and for runtime management.
This leaves 15 cores available for each application.
Each core is restricted to running only a single thread, as no context switching is supported, which leads to a possible number of threads between 1 and 15.
The core-composition is not used on the master core, leaving 15 physical cores to be distributed to each of the worker threads.
Cores can be fused together to form a logical core with up to 15 physical cores, making the total number of cores assigned to a thread between 1 and 15.
Of course, not all cores have to be assigned to a thread, in this case all remaining cores that aren't in a composition or a thread are turned off.
Overal, this leads to a total space size of 32,767 unique combination per benchmark as previously described in Section~\ref{sec:motivation}.

\subsection{Sample Space}


Given a partition, any benchmark that is split into 15 threads requires 32,767 executions to cover the entire space.
Running an exhaustive exploration of the space for a single benchmark requires approximately a week of simulation on a 572+ node supercomputer.
For this reason, a sample of 1,316 random points from the entire space is utilised.
This roughly corresponds to 100 core compositions for each number of threads; the actual number, 1,316 is smaller than 1,500 since for low and high thread counts there are less than 100 possible different core composition.
For example, a single thread can have only up to 15 different core-compositions (1 through 15) whilst 15 threads can only have a single core given to each thread.
\bench{InsertionSort} is the only exception since it can at most only be split into 5 threads leading to 415 sample points.
Overall, the space exploration required one week of simulation on the supercomputer~\cite{ecdf}.

\begin{figure}[t]
  \centering
    \includegraphics[width=1\textwidth]{streamit-paper/graphics/ESCProx.pdf}
    \caption{Statistical (plain line) and actual proximity (dotted line, this is only done for 5 benchmarks) to best performance using a subset of the sample space.}\label{fig:prox}
\end{figure}

%Define stopping criterion?
To gain confidence that the best configuration from the sample space is indeed close to the real best in the entire space, a statistical model based on the Early Stopping Criterion defined in~\cite{vuduc2003AutomaticPerf} is deployed.
This statistical model estimates, given a sample of the total space, if the best observed performance of that sample space is within a percentage of the statistical best performance, a more detailed explanation can be found in Chapter~\ref{chp:bg}.
The results demonstrate that the sample space selected is representative of the whole space.

Figure~\ref{fig:prox} shows, for each of the benchmarks, the proximity to the statistical best when increasing the sub-sample space given a maximal uncertainty of 5\%  (\ie minimum 95\% confidence).
As can be seen by the plain line, the model shows that the best sample point is actually within 5\% (0.05 proximity) of the best for all the benchmark.
The proximity is measured by taking the best currently observed point and comparing it to the latest discovered point.
To further prove that the statistical model based on the Stopping Criterion is indeed accurate, an exhaustive exploration of five benchmarks is conducted.
The dotted line in figure~\ref{fig:prox} shows the actual proximity to the best for \bench{Audiobeam}, \bench{Beamformer}, \bench{BitonicSort}, \bench{CFAR} and \bench{FMRadio}.
As can be seen after 1316 samples, the achieved performance is actually very similar to the one predicted by the statistical model, hence confirming prior work~\cite{vuduc2003AutomaticPerf}.
To summarize, it can be concluded that the best point found in the sample space of 1,316 points is at least within 5\% of the real best in the exhaustive space with 95\% confidence.

\subsection{Synthetic Benchmarks}

\begin{table}[t]
% The FFT need are variable
  \small
 \begin{tabular} { | l | l | l | }
 \hline
 & Av. Number of SplitJoins & Average Number of Filter Instances \\ \hline
 Selected Benchmarks & 14 & 52 \\ \hline 
 Synthetic & 22 & 64 \\ \hline
 \end{tabular}
 \caption{Data on the synthetic benchmarks compared to the selected benchmarks}~\label{tab:synthvsreal}
 \end{table}

One of the difficulties of building a machine learning based model for StreamIt is the lack of benchmarks available~\cite{wang2013partitionstreamit}.
To overcome this problem generating synthetic benchmarks can be a solution~\cite{cumminsopencl2017}.
Thus synthetic StreamIt benchmarks are generated and statistics are gathered from them in a similar style as in~\cite{wang2013partitionstreamit}.
In this chapter, the synthetic benchmarks are used to build a model for predicting the number of threads, which will be described later in section~\ref{sec:ml}

%Cite repository
To ensure that the synthetic benchmarks are representative of realistic benchmarks they are created using filters from a set of example StreamIt programs found in the example directory in the repository.
30 different possible filters with different incoming and outgoing rates and different inputs and outputs types are used to increase the variety of the dataset.
To ensure that the synthetic benchmarks are similar to real benchmarks, the total number of filters and split joins are within the average of the realistic benchmarks.

Table~\ref{tab:synthvsreal} gives the average number of SplitJoins and filter instances for the synthetic benchmarks vs the real benchmarks used in this chapter.
As can be seen, the synthetic benchmarks, on average, have more SplitJoins than the real benchmarks; this is due to the fact that a few of the benchmarks presented in the chapter don't have SplitJoins at all which can quickly reduce the average.
%Maybe say a bit more here.
Since these benchmarks are built to be used for predicting the number of threads an application should use, and SplitJoins are explicit declarations of task-level parallelism, having a higher average number of SplitJoins is not detrimental to building the model.


\section{Design Space Exploration}
This section describes the exploration of the software/hardware co-design space.
The software side includes partitioning the program, determining the number of threads and the specific source-level optimisations.
The hardware side is about finding out the best core composition that maximizes performance for a given partitioning.

\subsection{Thread Partitioning}

This section first starts with analyzing the impact of thread partitioning on performance.
In this section, the term optimal number of threads defines the number of threads which results in the best performance for any given benchmark.
Thread partitioning is about deciding how many threads to create and how to partition StreamIt filters into these threads.

To simplify this study, the default streaming partitioner is used to decide on how to allocate filters to threads which is based on simulated annealing~\cite{simulatedAnnealing1983}.
On the hardware side, two scenarios are considered: the ``without composition scenario'' where there is exactly one core per thread and the ``with composition scenario'' where each thread receives between 1 and 15 cores.
Figure~\ref{fig:threadtrend} shows how performance varies under both scenarios as a function of the number of threads.
In this figure, the ``with composition scenario`` uses points from the sample space that result in the fastest execution time for a given number of threads.
Regardless of how cores are composed it can be observed that curves for a benchmark follow the same trend.
As can be seen in Figure~\ref{fig:threadtrend}, the optimal number of threads using core composition is very similar to the scenario without composition as both curves follow the same performance trends.
This is due to the fact that StreamIt is oriented towards task-level parallelism and thus, multithreading will be a natural fit for performance improvements whilst core-composition may have less of an effect overall.
As Figure~\ref{fig:threadtrend} shows that the performance trends for both with and without composition are similar when it comes to thread counts, this  means that the optimal number of threads for a benchmark can be estimated independently from the hardware composition.
The system can therefore proceed in two stages: first determine the optimal number of threads and then decide on a core composition.

Figure~\ref{fig:threadtrend} also shows that the performance of most benchmarks starts to deteriorate passed a certain number of threads making it critical to not over-allocate threads.
This number of threads varies between benchmarks, thus it motivates the use of machine learning to decide the optimal number of threads to use.
Finally it is important to observe that executions without compositions always perform worse.
This demonstrates that composing cores is essential to obtain the best performance from a workload.


\subsection{Core Composition}

\begin{figure}[t]
  \includegraphics[width=1\textwidth]{streamit-paper/graphics/filterbank_tot.pdf}
  \caption{Distribution of FilterBank performance when modifying the amount of threads and compositions.}\label{fig:fbtotal}
\end{figure}

Using core composition, the processor fuses a number of cores and associates them to a thread to increase the thread's performance.
Whilst this flexibility is advantageous, choosing the right amount of cores for a given thread is difficult due to the large number of possible configurations~\cite{gulati2008multitaskingdmc}.

Figure~\ref{fig:fbtotal} shows how threading and core-compositions affect performance for the \bench{FilterBank} benchmark.
The curves represent the density distribution for different core compositions as a function of the number of threads.
The right hand side Y-axis represents the number of threads present in the current version of the benchmark whilst the left Y-Axis represents the density normalized by the total number of points in the design space.
For each of the threaded versions the benchmark runs using 100 different core-compositions.
The density curve for thread 15 is a single point as there exists only a single composition, so a line is drawn to represent where that point lies.

The width of each of the curves represents the influence of composition on the \bench{FilterBank}'s performance for a given number of threads.
For this benchmark, the impact of having core-composition enabled often leads to a 1.5x speedup compared to running only in multi-threaded mode; this can be seen for 1 to 4 threads.
Interestingly, as more threads are used, performance worsens, echoing the results shown in the previous section.
This is due to the fact that when the number of threads is increased, synchonization between threads will increase whilst the potential number of corse which can be fused decreases.
In the case where the application does not feature highly parallel tasks, de-prioritising core compositions can negatively impact performance.
This signifies that for the benchmark \bench{FilterBank}, it is more important to fuse cores with a small amount of threads rather than add more and more threads to the application.


\begin{figure}[t]
  \includegraphics[width=1\textwidth]{streamit-paper/graphics/filterbank_unroll.pdf}
  \caption{Distribution of FilterBank performance when modifying the amount of threads, composition and unrolling factor.}\label{fig:fbunroll}
\end{figure}


\subsection{Impact of Loop Transformation}
As seen in Chapter~\ref{chp:Background}, composing cores exploits block-level parallelism by running multiple EDGE blocks on a logical core.
As physical cores in a logical core must communicate to submit block address predictions, and commit information to each other, having a small number of blocks will reduce the communication overhead.
In a core-composition, commit information can be a core informing another core that it has become the non-speculative core, or that registers can be read or written to.
Since physical cores can fetch more than a single block when the blocks are made of a small number of instructions, if the program being executed is comprised of small blocks this will cause a composition to fetch a high amount of blocks.
Thus finding methods to increase the average size of the blocks can lead to reduced overhead.
One method of increasing the size of the blocks is through loop unrolling.
This section therefore analyses the impact of loop unrolling on the StreamIt benchmarks.

In this Chapter, unrolling is done at the source level via a flag passed to the StreamIt source-to-source compiler.
Given a number of times the loops must be unrolled, the StreamIt source-to-source compiler will generate the multi-threaded C++ code with the loops unrolled.
Figure~\ref{fig:fbunroll} presents an example of how loop unrolling affects performance on the \bench{FilterBank} benchmark.
The graph presents the same information as Figure~\ref{fig:fbtotal} but comparing .
Figure~\ref{fig:fbunroll} shows that unrolling loops for \bench{FilterBank} can improve performance by up to 1.42x compared to the fastest non-unrolled version.
Another observation is that the best execution times for each of the threaded versions when unrolling does not follow the same trend seen in Figure~\ref{fig:threadtrend}.
The leftmost curve performance peaks at two threads whereas the rightmost peaks at 3 compared to 4 in the non-unrolled version.

\begin{figure}[t]
  \includegraphics[width=1\textwidth]{streamit-paper/graphics/unrolling_vs_no_single_core.pdf}
  \caption{How unrolling affects how much performance can be obtained via core-composition on the single-threaded versions of each benchmarks.}\label{fig:unroll_summary}
\end{figure}

Figure~\ref{fig:unroll_summary} goes into more details on how unrolling affects the amount of speedup obtained by running each of the StreamIt benchmarks on a single thread using different number of cores in the composition.
In this figure, the X axis represents the number of cores in the composition, ranging from single core to 15.
The Y axis compares the execution time in number of cycles for the benchmark using a single core vs a given core-composition.
The colours of the lines represent with and without unrolling.
As can be seen, for the set of benchmarks used throughout this chapter, five benchmarks benefit from unrolling.
These benchmarks are \bench{Beamformer}, \bench{ChannelVocoder}, \bench{FFT6}, \bench{FilterBank} and \bench{FMRadio}. 

\begin{figure}[t]
  \includegraphics[width=1\textwidth]{streamit-paper/graphics/unroll_speed_bars.pdf}
  \caption{Speedup obtained .}\label{fig:unroll_bars}
\end{figure}
Figure~\ref{fig:unroll_bars} complements Figure~\ref{fig:unroll_summary} by showing the speedup obtained by unrolling loops when executing the benchmark on a single core.
On average, the information shown in Figure~\ref{fig:unroll_bars} coincides with the data shown in Figure~\ref{fig:unroll_summary}: benchmarks that don't scale also see no difference in performance when loop unrolling is called.
For the benchmarks that do not scale with unrolling; this is most certainly due to the for loops containing conditional statements which may keep the blocks size small.
When a loop that holds multiple conditional statements is unrolled, conditional statements may not be fused into a single block; thus the block size does not change.
Benchmark \bench{FMRadio} sees a 3x improvement compared to the non-unrolled version, this is due to the fact that all the loops are fully unrolled, reducing the total number of instructions required to execute the task.
For the \bench{FFT6} benchmark, unrolling loops will actually cause the single-core version to be slower than its not unrolled version.
%I think this is due to a refreshing performance thing
This is due to the fact that for \bench{FFT6}, the source to source unrolling adds intermediate variables in the loop which increase the number of loads and stores.
Whilst it may be slower on a single core, as seen in Figure~\ref{fig:unroll_bars}, having a core-composition running the thread will still result in faster execution that without loop-unrolling.

Figure~\ref{fig:unroll_size} shows the influence of loop unrolling on the average size of an EDGE block for each of the benchmarks.
The size represents the number of instructions exected in each of the EDGE blocks.
As can be seen, the data for \bench{Beamformer} and \bench{FFT6} in Figure~\ref{fig:unroll_size} corroborate with the idea that larger block sizes will result in better performance when fusing cores.
However whilst benchmarks \bench{ChannelVocoder}, \bench{FilterBank} and \bench{FMRadio} also see an increase in blocksize, it is not as important and averages out at a 1.22x increase.
%Be clearer here.
That said, even a small amount of increase can help improve the scalability of core-composition.

\begin{figure}[t]
  \includegraphics[width=1\textwidth]{streamit-paper/graphics/unrolling_size.pdf}
  \caption{Average size (in instructions) of blocks executed with and without unrolling for each benchmark .}\label{fig:unroll_size}
\end{figure}

Overall, this section has shown that loop unrolling can improve performance by increasing the size of blocks for the benchmarks which helps improve the efficiency of core-compositions.

\subsection{Co-Design Space Best Results}


This section  presents the results of the entire co-design space exploration.
Figure~\ref{fig:overviewhist} characterizes how much of a performance increase, over a baseline of a single-core single-thread with O2 optimisations, can be obtained with and without unrolling.
For each benchmark, the \textit{THREAD} bar represents the maximal speedup obtained by dividing the program into threads without fusing cores.
The \textit{CORE} bar represents the best speedup when the benchmark is executed in a single thread and fuse cores.
\textit{BOTH} represents the best speedup obtained for each benchmark using a combination of \textit{THREAD} and \textit{CORE}.
Finally, for each benchmark, the results are obtained for both an unrolled and not unrolled version to compare how the compiler optimisation affects performance.
Figure~\ref{fig:overviewhist} shows that when loops are not unrolled, composing cores will not greatly improve performance.
This is due to the fact that the amount of ILP found in filters without the unrolling is too little for there to be any benefit of composing cores.

In the scenario where there are no specific optimisations for composition, multithreading will be the main source of performance.
This can be seen when studying the geometric mean,without unrolling.
Finding the optimal number of threads gives a speedup of 1.92 compared to 1.33 when using only core composition, which is an improvement of 44\%.
This changes when taking unrolling into account as the core compositions can be used more efficiently.
In this case, the speedup obtained from only composing cores is only 13\% worse than using only threads.
For the \bench{FMRadio} benchmark, unrolling makes using only core-composition better than only using threads.
This information corroberates with the data seen in Figure~\ref{fig:unroll_summary}; it presents a unique case where the effect of core composition is important enough to change the dominant performance enhancer.
The performance increase obtained via the source-level loop unrolling via the compiler demonstrates that some modifications to the code must be done to ensure optimal use of the dynamic multicore processor.
%Thus loop unrolling demonstrates that the StreamIt programs must be modified to take advantage of the core composition.

Overall the results demonstrate that multi-threading is the prevalent leader of performance, even with unrolling turned on.
This is natural as StreamIt applications are naturally geared towards TLP as most programs have at least one SplitJoin as seen in the Table~\ref{tab:splitjoin} which gives the number of split-joins per benchmark.
Benchmarks with SplitJoins will naturally benefit from splitting the program into threads~\cite{thiesStreamit2010}.
%Make sure this is 100% true but as far as I remember this is the case
Those that do not feature SplitJoins can still be parallelised by splitting a Pipeline into multiple parts.
For example, benchmark \bench{FIR} features no SplitJoins, yet splitting the Pipeline in 2 will result in a 1.40x speedup.
However, it is important to note that whilst finding the optimal thread mapping may result in higher performance improvements than finding the optimal composition for a single thread, the best performance is always obtained through a combination of both optimizations.
For cases such as \bench{BeamFormer} the optimal pairing results in a 1.8x speedup compared to simply finding the best multit-threaded version.
On average, the optimal combination leads to a 1.5x performance increase compared to only multithreading.

\begin{landscape}
\begin{figure}\hspace{-1em}
    \includegraphics[width=1\linewidth,keepaspectratio]{streamit-paper/graphics/threadcompbench.pdf}
    \caption{Speedup obtained by choosing best core composition, best
      thread number and the combination of both optimisations. The baseline for the speedup measurement is single core, single thread execution using O2 compiler optimisations. Higher
      is better.}\label{fig:overviewhist}
\end{figure}
\end{landscape}
\subsection{Summary}
This section demonstrated that each parameter has a large effect on the performance of the workload.
Regardless of using core composition or not, there exists for each benchmark an optimal number of threads.
Unrolling is effective at exposing more opportunities for composition due to increased block sizes but there is a balance to strike between extracting large blocks and TLP.
Figure~\ref{fig:overviewhist} shows there is a 3x benefit (overall) by automating the partitioning of both the software (threads) and hardware (cores).



\section{Building a model for thread estimation}
% Small intro, re-explain that finding thread/core pairing is complicated and thus ML is a good idea.
As seen in the previous section, selecting the right number of threads and a good combination of cores is difficult.
This difficulty arises from trying to balance between exploiting larger composed cores with block speculation and ILP and between exploiting a larger number of logical cores via TLP.

The problem can be decomposed into two stages; first, determining the right number of threads and then selecting a good core composition.
In this section, two machine-learning models that predict the best thread partitioning and core composition to maximize performance are presented.

\subsection{Predicting the Best Number of Threads}

\paragraph{Synthetic Benchmark Generation}

One of the difficulties of building a machine learning based model for StreamIt is the lack of benchmarks available~\cite{wang2013partitionstreamit}.
Whilst there exists at least 30 realistic applications for StreamIt~\cite{theis2010empericalcharstreamit} this is simply not enough to create a large enough data set.
To overcome this problem synthetic StreamIt benchmarks are generated and gather statistics from them in a similar style as in~\cite{wang2013partitionstreamit}.
To ensure that the synthetic benchmarks are representative of realistic benchmarks they are created using filters from a set of micro-kernels found in some StreamIt examples.
30 different possible filters with different incoming and outgoing rates, different inputs and outputs are used.
To ensure that the synthetic benchmarks were similar to real benchmarks, the total number of filters and split joins are within the average of the realistic benchmarks.

For each generated application, 15 different threaded versions are generated.
Each of these versions is ran using a single core per thread and the cycle count is recorded.
This was repeated for 1000 unique randomly generated applications and record the best number of threads each time.

\paragraph{Extracting Features}

Once the benchmarks have been generated, the next step consists of gathering features for each applications.
In order to build the two machine learning models an initial set of over 50 features are extracted from StreamIt programs.
These features were extracted using pre-existing tools within StreamIt and some extra counters added by us.
The features selected for the models were determine through correlation analysis.
%In this section, when discussing correlation we specifically look at which variables correlate with the optimal number of threads.
In this section, variables which correlate with the optimal number of threads are explored.
These features are used by the model to make a prediction about the number of threads to use.

Figure~\ref{fig:corr} shows the 10 variables that correlate the most with the optimal thread number.
In StreamIt the term multiplicity references the number of times a filter will have to execute in a time slice when the graph is in a steady state~\cite{gordon2002streamcomp}.
In Figure~\ref{fig:corr} the highest correlating value, Number of Distinct Multiplicities, determines all different multiplicities found in the StreamIt graph.
Unconditionally executed blocks represent sets of operations in a filter that will always execute.

There are very little variables that highly correlate beyond Number of Distinct Multiplicities.
A high number of distinct multiplicities implies that subsets of filters will execute at different rates.
This means that certain filters may be local bottlenecks in a Pipeline for example.
When the number of distinct multiplicities is high this requires more threads to group filters with similar multiplicities.
The number of threads also depends on certain structural features such as Pipelines, SplitJoins and number of Filters.
Yet, these variables seem to hold less influence on the number of threads a program needs than the different multiplicities found in the graph.
This is most certainly due to the fact that whilst SplitJoins make parallelizable areas more visible, the amount of work contained in each stream of the SplitJoin, especially when this size is small, may actually make parallelizing the program worse due to ratio of communication to computation.

\begin{figure}
  \includegraphics[width=1\textwidth]{streamit-paper/graphics/corrGraph.pdf}
  \caption{The ten highest correlating features with the best number of threads for 1000 synthetic benchmarks.}\label{fig:corr}
\end{figure}
 
\paragraph{KNN Model}

A k-Nearest Neighbor (kNN) model was chosen to determine the number of threads to use for the application.
Given a new application to predict, the kNN classifier determines the $k$ closest generated applications in terms of the features.
The distance between the features is measured using the Euclidean for each application.
Once the set of $k$ nearest neighbors has been identified, the model simply averages the best number of threads for each of the $k$ nearest neighbors to make a prediction.
The parameter $k$ was determined experimentally using only the generated benchmarks.
A value of $k=7$ was found to lead to the best performance.

The features chosen are the variables displayed in Figure~\ref{fig:corr}.
Using cross validation is used to determine the efficiency by observing how close a classification is to the measured best thread number.
Using cross validation the model generated in this chapter has a 33\% accuracy of getting the predicted best thread number.
This increases to 57\% when allowing a prediction to be 1 thread away from the best and 67\% when 2 threads away.
Whilst the performance of pin-point accuracy is disappointing it does not incur more than a 12\% performance penalty when choosing a thread number which is +/- 1 from the best and 19\% when moving up to 2 threads away from the best.
This average is measured by looking at the thread performances without composition.

\begin{figure*}
  \center
  \includegraphics[width=1\textwidth]{streamit-paper/graphics/lineargraphs.pdf}
  \caption{Optimal number of cores in relation to the three highest correlating features. The maximum number of cores plateaus on the right hand side as this is the maximum possible amount.}\label{fig:maxav}
\end{figure*}



\begin{figure}[t]
  \includegraphics[width=1\textwidth]{streamit-paper/graphics/coreCorr.pdf}
  \caption{The ten highest correlating features with the optimal number of cores.}\label{fig:corrCore}
\end{figure}

\begin{figure}[t]
  \center
  \includegraphics[width=1\textwidth]{streamit-paper/graphics/lineargraphs.pdf}
  \caption{Optimal number of cores in relation to the three highest correlating features. The maximum number of cores plateaus on the right hand side as this is the maximum possible amount.}\label{fig:maxav}
\end{figure}

\subsection{Predicting Core Composition}

\subsection{Gathering Training Data}
Given that the optimal number of cores for a thread is independent of the number of threads found in the program, only the single threaded versions is used to determine the optimal number of cores.
For example, all benchmarks will only have a single core per thread when the application is partitioned in 15 threads as this is the maximum amount of cores that may be given to each thread rather than it being the optimal solution. 
Multiple versions of the benchmarks using different amounts of unrolling are included.
To determine the optimal number of cores only the training data that has a performance within 1\% of the best is selected. 

\subsection{Analyzing Features}

Figure~\ref{fig:corrCore} shows the highest correlating features with the optimal number of cores.
The features are very different from the ones presented in Figure~\ref{fig:corr} and overall there are higher correlating features.
The highest correlating value has a correlation factor of 0.88 which represents the number of operations found in a basic block of code.
The second feature is similar but only takes into account blocks that will be executed unconditionally, we have chosen to exclude blocks found in loops for this metric as there is still some form of condition for those blocks to be executed.
The next two feature compare the size of the average size of an unconditional block to the largest and smallest unconditional block.
The fifth feature measures the ratio of the number of unconditional blocks to conditional.

Overall there are no features distinct to StreamIt, such as pipelines or splitjoins that correlate highly with the optimal number of cores.
This is due to the fact that, from a single-threaded perspective, splitJoins and pipelines are less visible in terms of performance.
This is especially true of splitjoins as they will not be distributing data amongst different threads and, technically, a single-threaded StreamIt program is a long pipeline structure.
It can thus be infered that the optimal number of cores is independent of the structure of a StreamIt program.
Instead, it is more dependent on the amount of computation found in each program.

From Figure~\ref{fig:corrCore} the highest correlating features fit naturally under the assumptions that higher core compositions will perform better with larger blocks.
This is due to the fact that large blocks reduce the amount of branches predicted to populate all the cores with blocks which, in turn, reduces the latency of fetching blocks for all cores.
The necessity to correctly predict blocks to ensure that all cores are fully utilised explains why a higher number of unconditionally executed blocks compared to conditional blocks correlates highly.
This is once again due to reducing strain on branch prediction for higher core compositions.
StreamIt programs tend to not have a large quantity of conditional statements, thus the ratio of unconditional and conditional blocks is considered less important than the sizes of blocks.

Other features that were analysed included more fine-grained data on what types of operations were found in the blocks of code.
This involved finding ratios of floating point, integer and memory operations.
According to the correlation graph in Figure~\ref{fig:corrCore}, the composition of these blocks of code are not as important as their size or whether they are conditionally executed.

%Added text for thesis
%EDGE architecture's ability to fetch atomic instruction blocks and out-of-order execution encourages the focus on determining how much speculation is extracted from each filter.
%Unfortunately StreamIt programs do not tend to have a large quantity of conditional statements and when they do they tend to be quite small.
%This statement is reinforced by the correlation between the average number of conditional blocks with the optimal number of cores, which is only 0.2, compared to 0.809 for the average size of unconditional blocks.
%Thus there is no focus on using any speculative features from the StreamIt graph.

\subsection{Linear Regression Model}
Given that the optimal number of cores is highly correlated with a few features, a linear regressor is a natural choice to predict the best number of threads.
Figures~\ref{fig:maxav} represent how the first three highest correlating values affect the number of cores.
This figure was obtained by finding the best number of cores for a single threaded benchmark.
It is important to note that the top right corner points will always be flat as we can only allocate a maximum of 15 cores.



\section{Papers}

EDGE~\cite{edge_architecture,spdi}, TRIPS \cite{trips_cs}, E2
\cite{e2_arch}, Core Fusion~\cite{CoreFusion}, Sharing Architecture
\cite{sharing_arch} WiDGET~\cite{widget}, StreamIt \cite{stream_lang,
grid_streamit, machine_partitioning, empirical_stream_prog} and
mention other streaming languages~\cite{brook_gpu,wavescript_cmp}



\bibliographystyle{plain}
\bibliography{references}


\end{document}

