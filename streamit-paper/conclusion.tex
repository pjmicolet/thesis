This chapter introduced the problem of partitioning both software and hardware for a Dynamic Multicore Processor.
Given the ability to execute up to 15 threads, a DMP can have up to 32767 different configurations, making exhaustive search impossible.
Furthermore, restricting the space to scenarios such as only using multithreading or homogeneous logical cores will have an impact on performance.
Thus it is important to look at heterogeneous core-composition with multithreading to get the best performance.

In order quickly find a good configuration, machine learning is used and a set of 15 StreamIt benchmarks are explored.
By running each benchmark under 1300 different configurations, each benchmark was analysed to understand how much performance can be obtained through the best configuration.
This chapter showed that, on average, finding the optimal configuration and using the appropriate loop optimisations can lead to a performance increase of 3x compared to running the program on a single core.

Using the data gathered from the analysis, two models are created, one that predicts the number of threads via a k Nearest Neighbors model, whilst the other predicts the number of cores to fuse per thread using a linear regression.
The two models are able to predict configurations close to the performance of the best design points from the sampled space; on average within 16\% of the best.
This demonstrates that by analysing features of the programs, the decision of how to configure a dynamic multicore processor can be fully automated.