\subsection{Loop Unrolling}
In some cases, filters must be modified in order to better match with the potential hardware optimisations.
Filters containing a large amount of loops may result in blocks of operation that are too small to take advantage of fusing multiple cores.
Unrolling loops may also increase the size of hyperblocks as is decreases the amount of control flow.
For example, when potential vectorization may not easily be extracted from a loop unrolling it may yield similar results.
The StreamIt benchmarks often contain tightly knit loops that can be unrolled~\cite{}.
Determining the loop unroll factor can be difficult~\cite{} and depends on a varying set of factors.
One of the common constraints is on the stress loop unrolling can have on Instruction Caches by increasing the number of cache misses.
Fusing cores will allow us to reduce some of the penalties incurred by loop unrolling.
This is due to the E2's ability of allocating a greater amount of resources to a single-working-thread through cache fusion.

Figure X presents an example of how loop unrolling can affect performance on the \textbf{FMRadio} benchmark.
In this example we experimented using 3 different unroll factors: 2, 4, 8, 16, 32 and 64.
As we can see, the higher the unroll facter the greater the performance increase becomes.
This is due to the fact that by increasing the average block size we are able to fuse more cores together, resulting in a higher level of instruction level parallelism. 


