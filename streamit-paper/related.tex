\paragraph{Dynamic Multicore Processors}

DMPs such as CoreFusion~\cite{ipek2007CoreFusion} differentiate themselves to EDGE based DMPs on their Instruction Set Architecture (ISA).
CoreFusion uses a CISC/RISC based architecture which limits the degree of scalability (fusion), whereas EDGE based DMPs have shown promising scalability~\cite{kim2007composablelight, sibi2014}.
Other types of DMPs such as WidGET~\cite{Watanabe2010Widget} and Sharing Architecture~\cite{zhou2014sharingarch} present a fine-grain level of composition.
In these two architectures, cores can be created out of different components on the processor, including ALUs, floating point units and memory units.
This differs from CoreFusion and EDGE where a logical core is composed out of a set of physical cores.
This fine-grained composition can allow for even more optimisation but it increases the complexity of the problem.

\paragraph{Core Configuration}

Little work has been done on automatically determining the correct core composition for a given application.
The work conducted in~\cite{ipek2007CoreFusion,kim2007composablelight} manually configure their processors before running benchmarks.
In~\cite{santos2013nocdmc} they use information provided by the application to determine how to reconfigure some components of the processor.
This initial information then assists the rest of the reconfiguration, this process still requires input from the programmer though.
Therefore we present a novel method for automating the choice of core composition.  

\vspace{2mm}
\paragraph{Streaming Programming Languages}

There exist streaming languages that target different architectures.
For example Brook~\cite{buck2004brook} is designed to be used on GPUs and WaveScript for embedded systems~\cite{newton2008wavescript}.
These languages present different constructs to StreamIt, in particular they lack the graph oriented constructs. 
Lacking such constructs make these languages less attractive for tiled processors.

\paragraph{Partitioning StreamIt on multicore chip}

Previous work on scheduling streaming applications onto DMPs or heterogenous multicore chips focuses on finding mathematical ways of partitioning the graph onto the chip ~\cite{carpenter2009streammap,kudlur2008orchestratingstreamprog}.  
In Carpenter et al.'s work~\cite{carpenter2009streammap} they restrain themselves to partitioning a StreamIt application maintaining connectedness.
Connectedness can be defined as a subgraph where the filters are connected. 
This restriction reduces the number of potential partitions that can be generated by their algorithm and will put TLP in favour of ILP. 
Kudlur et al. in~\cite{kudlur2008orchestratingstreamprog} choose to represent the partitioning problem as an integer linear programming problem.
They start by fissionioning stateless filters to obtain the optimal load balance across all cores and assign the filters to a core using a modulo scheduler.
Farhad et al. also use integer linear programming in~\cite{farhad2012streamilp} to schedule StreamIt programs on multicore.
They profile the communication costs of the streaming programs by running the program using different multicore allocations and feed that information into their integer linear programming model.

\paragraph{Machine Learning} Using a machine learning model to partition StreamIt programs was previously explored in the work of Wang et al. in ~\cite{wang2013partitionstreamit}.
They use a k nearest neighbor model to determine the perfect partitioning of a StreamIt program for a multicore system. 
The features we extracted using correlation analysis are similar to those presented in the work of ~\cite{wang2013partitionstreamit}.
Unlike our work their model is used to find ways of fusing and fissioning filters to discover a new graph that can then be mapped onto a multicore system.

