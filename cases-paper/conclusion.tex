This chapter tackled the problem of dynamic reconfiguration of a Dynamic Multicore Processor at runtime.
Due to the fact that adding cores in a core-composition does not result in linear improvements, obtaining the fastest performance comes at the cost of energy.
Therefore, static core compositions are not an efficient method to speed up programs with phases.
Runtime dynamic reconfiguration of DMPs is therefore necessary to ensure that core compositions are used appropriately.

To better understand how core-composition is sensitive to branch prediction and block size, a limit study is conducted.
It shows that larger core-compositions favour large blocks as this reduces the strain on the branch predictor and also reduces the communication cost between cores.
To improve the size of blocks and block level parallelism, a set of compiler optisations such as loop inversion, loop unrolling and predication are discussed.

These optimistions are then applied on a set of vision benchmarks, and the performance of static core-compositions help show that these programs have phases of IPC patterns.
Using this information, two dynamic runtime reconfiguration schemes are created:  \textbf{DSpeed} that matches the speed of the fastest static core fusion and \textbf{DEff} that maximizes efficiency.
The chapter shows that \textbf{DSpeed} saves on average 42\% energy compared to the optimal static logical core whilst \textbf{DEff} can improve performance by up to 1.30x and reduce energy consumption by 1.20x on some benchmarks.

Finally a linear regression model is proposed to decide the number of cores to fuse at runtime for \textbf{DSpeed}.
This  model leads to a 37\% reduction in energy whilst maintaining the same level of performance as the optimal static scheme.

Overall, the contributions of this chapter are:

\begin{itemize}
\item Analysis of the limits of core fusion using an analytical model.
\vspace{-1em}
\item A study of the loop optimizations required to ensure efficient use of core fusion.
\vspace{-2.5em}
\item An in-depth comparison of static and dynamic core fusion schemes on the San Diego Vision Benchmark Suite.
\vspace{-1em}
\item A demonstration that core fusion has the potential to offer a large reduction in energy savings.
\vspace{-1em}
\item A demonstration that a simple linear-regression based model can predict the number of cores to fuse for different program phases.
\end{itemize}

%In this paper we have shown that whilst static core fusion already demonstrates promising results, it becomes harder to be efficient when increasing the size of logical cores.
%We explained theoretical limitations of static core fusion; without high branch prediction and large blocks, it under-performs.
%This was followed by a study of a suite of benchmarks, showing how performance varies greatly depending on the size of logical cores. 

%We then created two dynamic schemes: \textbf{DSpeed} that matches the speed of the fastest static core fusion and \textbf{DEff} that maximizes efficiency.
%Using these schemes we saw that \textbf{DSpeed} saves on average 42\% energy compared to the optimal static logical core for a given benchmark.
%We also showed that \textbf{DEff} can improve performance by up to 1.30x and reduce energy consumption by 1.20x on some benchmarks.
%Finally, we developed a simple linear regression model to decide on the number of cores to fuse at runtime to optimize for performance, leading to a 37\% reduction in energy while maintaining the same level of performance as a static scheme.


