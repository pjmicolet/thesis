The previous chapter explored how static ahead-of-time configuration of a dynamic multicore processor (DMP) can improve the performance of multi-threaded streaming applications and showed how a mix of core and thread partitioning leads to optimal performance.
Whilst static ahead-of-time configurations can improve the overall performance of programs, it cannot adapt itselt to phases a program may have.
A program phase can be used to describe multiple concepts, for example regions of code that exhibit different Instruction per cycle (IPC) performances, or variations of instruction mixes.
When programs exhibit different phases of IPC, this can result in situations where fusing multiple cores may be an inefficient solution, thus wasting resources.
This chapter therefore focuses on how a DMP can reconfigure itself at runtime to better match phases of a set of single-threaded applications.
Core composition is a technique that allows multiple cores to work on a single thread, allowing a DMP to improve the performance of single threaded applications.
As previously discussed the reconfiguration can happen ahead of time, especially when programs do not have a high variation in IPC.
However, programs can benefit from different adaptations of the hardware at different phases of its execution.
Some programs may have phases that feature high amounts of block-level parallelism whilst other phases do not perform any better on a set of composed cores.
Therefore, for programs that feature many phases, runtime reconfiguration is necessary to ensure efficient use of a DMP.

Whilst optimising for speed is an important method of evaluating a DMP, the ability to reconfigure the processor allows the hardware to adapt to different performance profiles.
For example, a DMP can reconfigure itself at runtime to maximise energy efficiency, instead of maximising speed.
Yet, whilst being able to adapt to different profiles is an attractive feature, determining when to reconfigure the processor is a challenging task for programmers.
It is also difficult for compilers to determine when to compose cores without profiling information as core composition is sensitive to branch prediction and memory dependencies that may only be determined at runtime.% relying on programmers to determine when to reconfigure the processor is a challenging task.
Automating the decision of when to compose cores allows the programmer to solely focus on the software rather than decide when to modify the hardware.

The chapter starts with an explanation of the theoretical limitations of core fusion and what can be expected in terms of performance.
Then the next section discusses the necessity of loop optimizations such as unrolling and loop interchange as they have a large impact on performance when fusing cores.
Using the San Diego Vision Benchmark Suite~\cite{sdvbs} (SD-VBS) as a use case, the chapter demonstrates that programs exhibit various phases with different amounts of BLP.
This is followed by a limit study on the potential for decreasing energy consumption while maintaining performance when adapting the number of cores for each program phase.
The results show that using dynamic core fusion can save up to 42\% on average while maintaining the same level of performance as a fixed number of cores.
The issue of latency introduced by reconfiguring the system is then discussed and its influence on the impact of core composition is explored.
Finally, a linear regression model that predicts the optimal number of cores per phase for reducing energy consumption while maintaining performance is generated.
This practical model leads to an average of 37\% saving in energy with no performance loss.

To summarise, the contributions are:

\begin{itemize}
\item Analysis of the limits of core fusion using an analytical model.
\vspace{-1em}
\item A study of the loop optimizations required to ensure efficient use of core fusion.
\vspace{-2.5em}
\item An in-depth comparison of static and dynamic core fusion schemes on the San Diego Vision Benchmark Suite.
\vspace{-1em}
\item A demonstration that core fusion has the potential to offer a large reduction in energy savings.
\vspace{-1em}
\item An analysis of how the overhead of reconfiguring the processor can affect the potential benefits of core composition.
\vspace{-1em}
\item A demonstration that a linear-regression based model can predict the number of cores to fuse for different program phases.
\end{itemize}
