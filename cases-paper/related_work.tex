
\paragraph{Reconfigurable Processors}

ElasticCore~\cite{tavanaElastic} proposes a morphable core that uses dynamic voltage and frequency scaling (DVFS) and microarchitectural modifications such as instruction bandwidth and capacity.
They propose a linear regressor model to determine reconfiguration, which uses more runtime information than ours, such as branch prediction and cache misses.
Overall Tavana et al's architecture is 30\% more energy efficient than a big.LITTLE architecture.

In~\cite{dubach13dynamic} they also propose a similar core architecture that modifies microarchitectural features.
They provide extensive analysis of SPEC 2000 benchmarks and demonstrate that machine learning and dynamic adaptation can double the energy/performance efficiency compared to a static configuration.

MorphCore~\cite{khubaibMorphCore2012} focuses on reconfiguring a core for thread level parallelism.
It switches between out-of-order (OoO) when running single threaded applications and an in-order core optimised for simultaneous multi threading (SMT) workloads.
This provides an opposite solution to our DMP: providing a large core made for ILP that can be modified to better fit TLP workloads.
MorphCore outperform a 2-Way SMT OoO core by 10\% whilst being 22\% more efficient.

All these projects focus on uni-core modifications, and traditional CISC/RISC like architecture which differs from our work.

\vspace{-0.5em}
\paragraph{Dynamic Multicore Processors}
Previous work on Dynamic Multicore Processors includes CoreFusion~\cite{ipek2007CoreFusion} and Bahurupi~\cite{pricopi2012bahurupi,pricopiSchedCoreComp2014}.
These architectures use a standard ISA and either fetch fixed sized instruction windows~\cite{ipek2007CoreFusion} or entire basic blocks~\cite{pricopi2012bahurupi}.
Other DMPs such as TFlex~\cite{kim2007tflex} and E2~\cite{e2} use an hybrid-dataflow EDGE ISA~\cite{burger04edge}. 
In TFlex, instructions from a block are executed on different fused cores.
In E2, a block is mapped to a fused core and all instructions from that block execute locally.

\vspace{-0.5em}
\paragraph{Dynamic Core Fusion}
In the work of Pricopi et al.~\cite{pricopiSchedCoreComp2014}, they show how dynamic reconfiguration is beneficial when it comes to scheduling tasks.
However, they do not discuss any method of automatically deciding the optimal configuration beyond a 4 core fusion.
Instead they use speedup functions determined from profile executions of applications to determine how to schedule tasks.
They do not discuss what software characteristics help determine when to reconfigure the cores, or how to optimise software.

Work on using machine learning to automatically choose a composition was achieved in~\cite{micolet2016dmpstream}.
This work does not involve changing the core fusion dynamically during the execution of the benchmark.
The machine learning model focuses on using high-level information from StreamIt's~\cite{thiesStreamit2010} language constructs.

\vspace{-0.5em}
\paragraph{Voltage Scaling}
Voltage scaling is another method of reducing energy consumption~\cite{paganiEECHM2017}, however this approach is orthogonal to DMPs~\cite{sibi}.
Whilst both methods adapt to programs phases, DMPs can also be used to speed up the execution of programs.
