\chapter{Introduction}
%From StreamIt paper
Multicore processors are now common in all computing systems ranging from mobile devices to data centers.
As advances in single threaded performance have slowed, multicore processors have offered a way to use the increasing numbers of transistors available.
However, designing processors that scale to a large number of cores is difficult and a shift towards tiled architecture seems inevitable.
A tiled architecture such as Tilera~\cite{bell2008tile} or Raw~\cite{waingold1997raw} is composed of smaller simpler cores that are placed on a regular grid.
This improves hardware scalability and enables multi-threaded applications to exploit the large core count.

% Tiled architecture problem: cores too weak => need reconfiguration
However, workloads that require high single threaded performance are penalized by the simple nature of each core~\cite{eyerman2010amdahl}.
One solution to this problem is heterogeneous multicores which utilize cores with different levels of power and performance.
Although heterogeneous multicores are common place in mobile devices, they have little reconfiguration or adaptive capabilities (\eg only two type of cores available for ARM big.LITTLE).
Dynamic multicore processors offer a solution to this problem by allowing cores to compose (or fuse) together~\cite{ipek2007CoreFusion} into larger logical cores to accelerate single threads.
This produces ``on-demand'' heterogeneity where cores are grouped to adapt to the workload's demand.
