\chapter{Setup}\label{chp:setup}

\begin{table}[ht]
\centering
\begin{singlespace}
\begin{tabular} { p{5cm}  p{5cm} }
      \toprule
      \textbf{Parameter} & \textbf{Values} \\ \midrule
	  Issue Width & 4  \\
	  Number of Lanes & 4 \\
      L1D cache size & 32kB \\
      L1I cache size & 32kB \\
	  L2 cache size & 2MB \\
	  \# of MSHR & 8 \\
	  LSQ Organisation & Out of Order \\
	  
	  \end{tabular}
	  \end{singlespace}
  \caption{Hardware characteristics of a single core of the processor.}
  \label{tab:processor}
\end{table}

\subsection{Dynamic Multicore Processor Simulator}

To evaluate the work a customizable cycle-level simulator for the EDGE architecture is used.
The simulator is verified against RTL implementation of an EDGE core and is within 5\% from that implementation.
This validation is done by running workloads on RTL and comparing the traces cycle-by-cycle with the software simulator.

To maintain a homogeneous view of the system, the same core configuration was used throughout the thesis.
The features of the core can be found in table~\ref{tab:processor}, and the processor is composed of 16 cores connected via a mesh network, with the L2 Cache being the shared last level cache.
The number of lanes represents how many blocks a single core can hold in its instruction window at a time, and is chosen to be 4 similar to the original work on the E2 EDGE processor~\cite{putnam2010e2}.
Whilst no current embedded processor on the market features 16 cores, it is not outside the realm of possibility, as state-of-the art flagship system on chips (SoCs) already feature up to 8 cores.
Also, previous studies on dynamic multicore processors for EDGE have explored processors with up to 32 cores~\cite{kim2007tflex, gulati2008multitaskingdmc}.
Having 16 cores available increases the number of possible configurations of the processor which in turn allows for a more impactfull exploration of core composition.

\subsection{Compiler}

All the benchmarks explored in this thesis are compiled using a closed-source EDGE compiler provided by Microsoft.
The benchmarks are compiled with \textit{-O2} optimisations as this is the highest level of optimisations available, hyperblock formation is also turned on.